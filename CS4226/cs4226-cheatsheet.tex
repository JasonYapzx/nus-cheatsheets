\documentclass[10pt,landscape]{article}
\usepackage{amssymb,amsmath,amsthm,amsfonts}
\usepackage{multicol,multirow}
\usepackage{calc}
\usepackage{ifthen}
\usepackage{graphicx}
\usepackage{xcolor}
\usepackage[utf8]{inputenc}
\usepackage{enumitem}
\usepackage{listings} 
\usepackage[landscape]{geometry}
\usepackage[colorlinks=true,citecolor=blue,linkcolor=blue]{hyperref}
\usepackage{fancyhdr}
\usepackage{tabularx}
\usepackage{lmodern}
\usepackage{soul} 

\lstset{
    tabsize=2,    
%   rulecolor=,
    language={java},
        captionpos = t,
        basicstyle = \scriptsize\ttfamily,
        frame=lines,
        numbersep=5pt,
        numbers=left,
        numberstyle=\scriptsize,
        backgroundcolor=\color{white},
        columns=fixed,
        extendedchars=false,
        breaklines=true,
        prebreak = \raisebox{0ex}[0ex][0ex]{\ensuremath{\hookleftarrow}},
        frame=single,
        showtabs=false,
        showspaces=false,
        showstringspaces=false,
        keywordstyle=\color[rgb]{0,0,1},
        keywordstyle=[2]\color{gray},
        commentstyle=\color{teal},
        stringstyle=\color{red},
        numberstyle=\color[rgb]{0.205, 0.142, 0.73},
}

\ifthenelse{\lengthtest { \paperwidth = 11in}}
    { \geometry{top=.20in,left=.20in,right=.20in,bottom=.20in} }
	{\ifthenelse{ \lengthtest{ \paperwidth = 297mm}}
		{\geometry{top=1cm,left=1cm,right=1cm,bottom=1cm} }
		{\geometry{top=1cm,left=1cm,right=1cm,bottom=1cm} }
	}
\pagestyle{empty}
\makeatletter
\renewcommand{\section}{\@startsection{section}{1}{0mm}%
                                {-1ex plus -.5ex minus -.2ex}%
                                {0.5ex plus .2ex}%x
                                {\normalfont\large\bfseries}}
\renewcommand{\subsection}{\@startsection{subsection}{2}{0mm}%
                                {-1explus -.5ex minus -.2ex}%
                                {0.5ex plus .2ex}%
                                {\normalfont\normalsize\bfseries}}
\renewcommand{\subsubsection}{\@startsection{subsubsection}{3}{0mm}%
                                {-1ex plus -.5ex minus -.2ex}%
                                {1ex plus .2ex}%
                                {\normalfont\small\bfseries}}
\newcommand{\subsubsubsection}{\@startsection{subsubsection}{3}{0mm}%
                                {-1ex plus -.5ex minus -.2ex}%
                                {1ex plus .2ex}%
                                {\normalfont\scriptsize\bfseries}}
\makeatother
\setcounter{secnumdepth}{0}
\setlength{\parindent}{0pt}
\setlength{\parskip}{0pt plus 0.5ex}

\renewcommand{\familydefault}{\sfdefault}
\renewcommand\rmdefault{\sfdefault}

\title{CS4226-cheatsheet}
% -----------------------------------------------------------------------

\begin{document}

\raggedright
\scriptsize


\begin{multicols*}{3}
\setlength{\premulticols}{0.1pt}
\setlength{\postmulticols}{0.1pt}
\setlength{\multicolsep}{0.1pt}
\setlength{\columnsep}{0.1pt}
\begin{tiny}
    \small{\textbf{CS4226 Cheatsheet AY24/25 || \href{https://github.com/JasonYapzx}{@JasonYapzx}}} \\
\end{tiny}

\subsection*{1. Performance}
\subsubsection*{Network Performance Metrics}
\begin{itemize}[topsep=0pt,noitemsep,wide=0pt, leftmargin=\dimexpr\labelwidth + 2\labelsep\relax]
    \item \textbf{Link Rate/Bandwidth/Capacity:}
    \begin{itemize}[topsep=0pt,noitemsep,wide=0pt, leftmargin=\dimexpr\labelwidth + 2\labelsep\relax]
        \item Cable - 1~100 Mbps, Fiber optics - 1~100 Gbps
        \item Ethernet - 3 Mbps to 100 Gbps
        \item How many bits can be `pushed' onto a link per unit time
        \item How fast we can place bits onto the \underline{physical medium} delivering
    \end{itemize}
    \item \textbf{Throughput}
    \begin{itemize}[topsep=0pt,noitemsep,wide=0pt, leftmargin=\dimexpr\labelwidth + 2\labelsep\relax]
        \item Throughput of a TCP/UDP flow
        \item How many bits can be communicated per unit time
        \item How fast the bits get delivered through the route (regardless of the medium)
    \end{itemize}
    \item \textbf{End-to-end delay}
    \begin{itemize}[topsep=0pt,noitemsep,wide=0pt, leftmargin=\dimexpr\labelwidth + 2\labelsep\relax]
        \item Components
        \begin{itemize}[topsep=0pt,noitemsep,wide=0pt, leftmargin=\dimexpr\labelwidth + 2\labelsep\relax]
            \item Processing + Queueing Delay + Transmission + Propagation
        \end{itemize}
        \item Related: bandwidth, packet and queue size, distance, prop. speed
        \item Source $\rightarrow$ Destination
    \end{itemize}
    \item \textbf{Response Time | Round Trip Time}
    \begin{itemize}[topsep=0pt,noitemsep,wide=0pt, leftmargin=\dimexpr\labelwidth + 2\labelsep\relax]
      \item Source $\rightleftarrows$ Destination
    \end{itemize}
    \item \textbf{Others}
    \begin{itemize}[topsep=0pt,noitemsep,wide=0pt, leftmargin=\dimexpr\labelwidth + 2\labelsep\relax]
      \item Dropping probability
      \item Utilization (percentage of time link is busy)
    \end{itemize}
\end{itemize}

\subsubsection*{Packet Switching}
Compared to \textit{circuit-switching}, it allows more users to use the network at once.
\begin{itemize}[topsep=0pt,noitemsep,wide=0pt, leftmargin=\dimexpr\labelwidth + 2\labelsep\relax]
  \item Great for async bursty data
  \begin{itemize}[topsep=0pt,noitemsep,wide=0pt, leftmargin=\dimexpr\labelwidth + 2\labelsep\relax]
    \item resource sharing | simpler, no call setup
  \end{itemize}
  \item \textbf{excessive congestionn:} packet delay and loss
  \begin{itemize}[topsep=0pt,noitemsep,wide=0pt, leftmargin=\dimexpr\labelwidth + 2\labelsep\relax]
    \item protocols needed for reliable data transfer, congestion control
  \end{itemize}
  \item reserved resources (circuit) vs on-demand allocation (packet)
\end{itemize}

\subsubsection*{Packet-switching store-and-forward}
Entire packet must arrive at router before it can be transmitted onto the next link
$Delay = \frac{L}{R}$, if $L$ = packet size and link rate $R$

\subsubsection*{Statistical Multiplexing}
\begin{itemize}[topsep=0pt,noitemsep,wide=0pt, leftmargin=\dimexpr\labelwidth + 2\labelsep\relax]
  \item Sequence of A \& B has no fixed timing pattern
  \begin{itemize}[topsep=0pt,noitemsep,wide=0pt, leftmargin=\dimexpr\labelwidth + 2\labelsep\relax]
    \item Bandwidth shared on demand: \textit{statistical multiplexing}
  \end{itemize}
  \item Time Divison M.: each host gets same slot in revolving TDM frame
\end{itemize}

\includegraphics*[width=8.5cm, height=4.4cm]{images/singleserverqueueing.png}

\subsubsection*{Little's Law: $L = \lambda \times W$}
\begin{itemize}[topsep=0pt,noitemsep,wide=0pt, leftmargin=\dimexpr\labelwidth + 2\labelsep\relax]
  \item Arrival rate into the system $\lambda$:
  \begin{itemize}[topsep=0pt,noitemsep,wide=0pt, leftmargin=\dimexpr\labelwidth + 2\labelsep\relax]
    \item $\lambda \overset{def}{=} \lim_{t \rightarrow \infty} \frac{N(t)}{t}$
  \end{itemize}
  \item Average sojourn time (time $W_i$ for $i^{th}$ customer spent in) $W$:
  \begin{itemize}[topsep=0pt,noitemsep,wide=0pt, leftmargin=\dimexpr\labelwidth + 2\labelsep\relax]
    \item $W \overset{def}{=} \lim_{n \rightarrow \infty} \frac{1}{n} \sum_{j=1}^{1} W_j$
  \end{itemize}
  \item Time average\# of customers in system $L$:
  \begin{itemize}[topsep=0pt,noitemsep,wide=0pt, leftmargin=\dimexpr\labelwidth + 2\labelsep\relax]
    \item $L \overset{def}{=} \lim_{t \rightarrow \infty} \frac{1}{t} \int_{0}^{t} L(s) ds$
  \end{itemize}
  \item Avg \# of customers in system = arrival rate $\times$ avg. sojourn time
\end{itemize}

\subsection{2. Network Queueing Models}
\subsubsection*{Model packet flow: Arrival Pattern}
\begin{itemize}[topsep=0pt,noitemsep,wide=0pt, leftmargin=\dimexpr\labelwidth + 2\labelsep\relax]
  \item $T_i$ are independent and identically distributed \textbf{random variable}
  \item Inter-arrival time: $T_i \overset{\triangle}{=} t_{i+1} - t_i$
\end{itemize}

\subsubsubsection*{Random Experiment}
Consider a random experiment whose outcome cannot be determined in advance
\begin{itemize}[topsep=0pt,noitemsep,wide=0pt, leftmargin=\dimexpr\labelwidth + 2\labelsep\relax]
  \item \textbf{Sample space} $S$: set of all outcomes
  \item \textbf{Event} $E$: $\subset$ of sample space, event $E$ occured if outcome $s \in E$
  \item \textbf{Probability} function $P(E)$
  \begin{itemize}[topsep=0pt,noitemsep,wide=0pt, leftmargin=\dimexpr\labelwidth + 2\labelsep\relax]
    \item \hl{$0 \leq P(E) \leq 1$, $P(S) = 1 \Longrightarrow P(\bigcup^\infty_{i=1}E_i) = \sum^\infty_{i=1} P(E_i)$}
    \item For any sequence of events $E_1, E_2 \dots$ that are \textit{mutually exclusive}
  \end{itemize}
\end{itemize}

\subsubsubsection{Random Variable}
\begin{itemize}[topsep=0pt,noitemsep,wide=0pt, leftmargin=\dimexpr\labelwidth + 2\labelsep\relax]
  \item \textbf{Random variable}: function assigning real value to each outcome $s \in S$: For any set of \hl{$A \subseteq \mathbb{R}$: $P\{X \in A\} \overset{def}{=} P(X^{-1}(A))$}
  \item \textbf{Distribution function} $F$ of r.v. $X$ is defined on any \textit{real number} $x$ by: \hl{$F(x) \overset{def}{=} P\{X \leq x\} = P(X^{-1}((-\infty, x]))$}
  \item A random variable is \textit{continuous} if there exists a \textbf{probability density function} (pdf) $f(x)$ such that: \hl{$f(s) \overset{def}{=} \frac{d}{dx}F(x)$}
  \item 2 random variables are \textit{independent} if realization of 1 does not affect probability distribution of other: $f_{X,Y}(x,y) = f_X(x)f_Y(x)$
  \item \textbf{Expectation or mean} of random variable $X$: \\ \hl{$E[X] \overset{def}{=} \int_{-\infty}^{+\infty}xf(x)dx$ \textbf{\textit{or}} $E[X] \overset{def}{=} \sum_{x=-\infty}^{+\infty} xP\{X=x\}$}
\end{itemize}

\subsubsubsection{Exponential Distribution}
Continuous random variable $T$ follows/has an \textbf{exponential distribution} with parameter $\lambda > 0$ if for $x \geq 0$, where $\frac{1}{\lambda}$ is the frequency
\begin{itemize}[topsep=0pt,noitemsep,wide=0pt, leftmargin=\dimexpr\labelwidth + 2\labelsep\relax]
  \item \hl{$F(x) = P\{T \leq x\} = 1 - e^{-\lambda x}$ or $\overset{\_}{F}(x) = P\{T > x\} = e^{-\lambda x}$}
  \item \hl{$f(x) = \frac{dF(x)}{dx} = \lambda e^{-\lambda x}$}
  \item Average: $E[T] = \int_{-\infty}^{\infty} xf(x) dx = \frac{1}{\lambda}$ $\Longrightarrow$ inter-arrival rate
  \item \textbf{Memoryless property:} $P\{T > s + t | T > s\} = P\{T > t\}$
\end{itemize}

\includegraphics*[width=8.5cm, height=4.6cm]{images/poissonmerge.jpg}

\subsubsection*{Model packet flow: Service Time}
\begin{itemize}[topsep=0pt,noitemsep,wide=0pt, leftmargin=\dimexpr\labelwidth + 2\labelsep\relax]
  \item Packets have varying length
  \begin{itemize}[topsep=0pt,noitemsep,wide=0pt, leftmargin=\dimexpr\labelwidth + 2\labelsep\relax]
    \item takes different amt of time to process, same packet given different link capacity/rate
    OR packets with different lengths under fixed links
  \end{itemize}
  \item Service time $S_i$
  \begin{itemize}[topsep=0pt,noitemsep,wide=0pt, leftmargin=\dimexpr\labelwidth + 2\labelsep\relax]
    \item processing time of packet $i$ under fixed link rate
    \item follows i.i.d r.v. $S$ with mean $E[S] = \frac{1}{\mu}$
  \end{itemize}
\end{itemize}

\subsubsubsection{M/M/1 Model}
\begin{itemize}[topsep=0pt,noitemsep,wide=0pt, leftmargin=\dimexpr\labelwidth + 2\labelsep\relax]
  \item single server with queue of $\infty$ size
  \item poisson arrival with rate $\lambda$
  \item exponential i.i.d service time with rate $\mu$
  \item arrival/service times independent FIFO service discipline
\end{itemize}
\includegraphics*[width=7.5cm, height=1cm]{images/m1model.jpg}

\subsubsection*{Result}
\begin{itemize}[topsep=0pt,noitemsep,wide=0pt, leftmargin=\dimexpr\labelwidth + 2\labelsep\relax]
  \item Utilization $\rho$: percentage of time that \textit{server} is busy OR
  \item probability of random observation finds server busy
  \item general result (applies but not limited to M/M/1 Model)
  \item $\rho = \frac{\lambda}{\mu}$ need condition for $\lambda < \mu$ for system stability
\end{itemize}

\subsubsubsection{Main Result (without proof)}
\begin{itemize}[topsep=0pt,noitemsep,wide=0pt, leftmargin=\dimexpr\labelwidth + 2\labelsep\relax]
  \item \hl{$\pi_i$} $ = \%$ of time exactly $i$ packets or customers in system $\rightarrow$ server + queue
  \item $P\{L=i\}$ that random observation finds $i$ packets in the system
  \item For M/M/1 system, we have $\pi_i = P\{L=i\} = p^i(1-p)$
  \begin{itemize}[topsep=0pt,noitemsep,wide=0pt, leftmargin=\dimexpr\labelwidth + 2\labelsep\relax]
    \item \hl{$\pi_0 = \rho^0(1-\rho) \Rightarrow 1 - \rho$ $\rightarrow \%$} of time server idle/no packets in system
    \item Discrete distribution $\rightarrow$ density function $= P\{L = i\} = \rho(1-\rho)^i$
    \item Follows a geometric distribution
  \end{itemize}
\end{itemize}


\begin{itemize}[topsep=0pt,noitemsep,wide=0pt, leftmargin=\dimexpr\labelwidth + 2\labelsep\relax]
  \item \textbf{System + Queue}
    \begin{itemize}[topsep=0pt,noitemsep,wide=0pt, leftmargin=\dimexpr\labelwidth + 2\labelsep\relax]
      \item Average no. of packets in system \hl{$E[L] = \frac{\rho}{1-\rho}$}
      \item Average sojourn time of packets \hl{$E[W] = \frac{1}{\mu - \lambda}$}
    \end{itemize}
  \item \textbf{Queue only}
    \begin{itemize}[topsep=0pt,noitemsep,wide=0pt, leftmargin=\dimexpr\labelwidth + 2\labelsep\relax]
      \item Average no. of packets in the queue \hl{$E[Q] = \frac{\rho^2}{1-\rho}$}
      \item Average queueing delay of packets \hl{$E[D] = E[W] - \frac{1}{\mu}$}
    \end{itemize}
\end{itemize}

When $\lambda = \mu$ it is unstable, $E[L], E[W] \rightarrow \infty$, because of the randomness.  
\subsubsubsection{Throughput of system}
\begin{itemize}[topsep=0pt,noitemsep,wide=0pt, leftmargin=\dimexpr\labelwidth + 2\labelsep\relax]
  \item General system, throughput is $\min(\mu, \lambda)$
  \item Stable system $\mu < \lambda$: accept more packets into the system then service them, thus it will be bounded by $\lambda$
  \item Throughput \& queueing delay \textbf{positively correlated} if $\lambda$ constant
\end{itemize}

\subsubsection*{Effective Bandwidth}
\begin{itemize}[topsep=0pt,noitemsep,wide=0pt, leftmargin=\dimexpr\labelwidth + 2\labelsep\relax]
  \item \textbf{Phyiscal link capacity:} theoretical processing limit of hardware
  \item \textbf{Effective bandwidth of the link:} actual throughput that can be achieve / quality of service
\end{itemize}

\subsubsection*{Statistical Multiplexing vs TDM}
\includegraphics*[width=8.5cm, height=1.5cm]{images/statisticalmultiplexingvstdm.png}
\begin{itemize}[topsep=0pt,noitemsep,wide=0pt, leftmargin=\dimexpr\labelwidth + 2\labelsep\relax]
  \item Each Poisson stream its own queue $(\frac{\lambda}{k} , \frac{\mu}{k})$ or shared queue $(\lambda, \mu)$
  \item \hl{$\rho = \frac{\lambda}{\mu} = \frac{\lambda / k}{\mu / k}$} $\Longrightarrow$ same server utilization
  \item \hl{$E[W] = \frac{1}{\mu} \frac{1}{1-\rho}$} $\Longrightarrow$ avg. sojourn time for each user $\uparrow$ by $\times k$ under TDM
  \item \hl{$E[Q] = \frac{\rho^2}{1-\rho}, E[L] = \frac{\rho}{1-\rho}$} $\Longrightarrow$ avg. no. of waiting + total users are also $k$ times larger
\end{itemize}

\subsubsection*{Burke's Theorem}
\includegraphics*[width=8.5cm, height=1cm]{images/burketheorem.png}
If M/M/1 system with arrival rate $\lambda$ starts in steady state
\begin{itemize}[topsep=0pt,noitemsep,wide=0pt, leftmargin=\dimexpr\labelwidth + 2\labelsep\relax]
  \item departure process is \textit{Poisson} with rate $\lambda$
  \item number of customers in system at any time $t$ is \textit{independent of sequence} of departure times prior to time $t$
\end{itemize}

\subsubsubsection{Tandem Queues}
\begin{itemize}[topsep=0pt,noitemsep,wide=0pt, leftmargin=\dimexpr\labelwidth + 2\labelsep\relax]
  \item Utilization of each server $i$ becomes $p_i = \frac{\rho}{\mu_i}$
  \item By independence, joint probability: $P\{L_1=j, L_2=k\} = P\{L_1 = j\} \cdot P\{L_2 = k\} = \rho_1^j(1-\rho_1)\rho_2^k(1-\rho_2)$
\end{itemize}

\subsubsubsection{Acyclic Network with Probabilistic Routing}
\begin{multicols*}{2}
\includegraphics*[width=4.1cm, height=1.8cm]{images/acyclicnetworkwithprob.png}
\columnbreak 
\begin{itemize}[topsep=0pt,noitemsep,wide=0pt, leftmargin=\dimexpr\labelwidth + 2\labelsep\relax]
  \item Diff. flows may take different paths. 
  \item Each link shared by different flows
\end{itemize}
\end{multicols*}

\subsubsubsection{Jackson Network (Feedback Loop)}
\begin{multicols*}{2}
  \includegraphics*[width=4.1cm, height=1.8cm]{images/jacksonnetwork.png}
  \columnbreak 
  \begin{itemize}[topsep=0pt,noitemsep,wide=0pt, leftmargin=\dimexpr\labelwidth + 2\labelsep\relax]
    \item $P_{ij}$: join queue that leaves system
    \item $P_{ik}$: Probability to join another system that will be a feedback loop to rejoin system
  \end{itemize}
\end{multicols*}


\textbf{Solving the Jackson Network}
\begin{itemize}[topsep=0pt,noitemsep,wide=0pt, leftmargin=\dimexpr\labelwidth + 2\labelsep\relax]
  \item Effective arrival rate $\lambda_i$:
  \begin{itemize}[topsep=0pt,noitemsep,wide=-9pt, leftmargin=\dimexpr\labelwidth + 2\labelsep\relax]
    \item outside arrival directly to server $i$, i.e. $r_i$ + feedback arrivals (other servers)
    \item (stable system, leaving system rate = arrival rate $\approx lambda_j$)
    \item $\lambda_i = r_i + \sum^{n}_{j=1}\lambda_j P_{ji}$
    \item \textbf{Matrix form:} $\lambda = r + \lambda P \Rightarrow \lambda = r(I - P)^{-1}$
  \end{itemize}
\end{itemize}

\subsection{3. Resource Allocation}
\begin{itemize}[topsep=0pt,noitemsep,wide=0pt, leftmargin=\dimexpr\labelwidth + 2\labelsep\relax]
  \item flow 1 rate $\lambda_1$, flow 2 $\lambda_2$, link capacity $\mu$ packets/second respectively
  \item divide capacity into $\mu = \mu_1 + \mu_2$, serving 2 flows separately
  \item packet flows might have different \textit{average delay}
  \item given delay gurantee, acheived throughput: $\lambda = \mu - \frac{1}{E[W]}$
\end{itemize}

\subsubsubsection{Motivation}
\begin{multicols*}{2}
  \includegraphics*[width=4cm, height=2.6cm]{images/utility.png}
  \columnbreak 
  \begin{itemize}[topsep=0pt,noitemsep,wide=0pt, leftmargin=\dimexpr\labelwidth + 2\labelsep\relax]
    \item internet is free, suitable for elastic services: \verb|Telnet|, \verb|FTP|, \verb|DNS|m \verb|SMTP|
    \item not suitable for real-time applications: VoIP, Skype, Zoom
    \item allocate different amt of resources to different application flows
    \item $U_i$ (user's happiness) as a function of performance metrics 
  \end{itemize}
\end{multicols*}

\subsubsubsection{Fairness}
\begin{itemize}[topsep=0pt,noitemsep,wide=0pt, leftmargin=\dimexpr\labelwidth + 2\labelsep\relax]
  \item equal share of resources: split allocation equally based on available resources
  \begin{itemize}[topsep=0pt,noitemsep,wide=0pt, leftmargin=\dimexpr\labelwidth + 2\labelsep\relax]
    \item Equal share amongst all:
    Demand $\boldsymbol{d} = (d_1, d_2, d_3, d_4) = (3, 4, 5, 6)$ \\
    Allocate $\boldsymbol{x} = (x_1, x_2, x_3, x_4) = (2.5, 2.5, 2.5, 2.5)$
    \item \textbf{\textit{Fulfill small demands completely (equal share higher demands):}}
    Demand $d = (1, 4, 5, 6)$ Allocate $x = (1, 3, 3, 3)$ ($d_1$ fulfilled)
  \end{itemize}
  \item \textbf{feasible solution}
  \begin{itemize}[topsep=0pt,noitemsep,wide=0pt, leftmargin=\dimexpr\labelwidth + 2\labelsep\relax]
    \item for links with capacity $C_1, C_2, C_3$, and allocation $x = (x_1, x_2, x_3)$
    \item $0 \leq x_i \leq d_i$ for all $i$, where flow demand = $d_i$
    \item $x_1 + x_2 \leq C_1$; etc. (the flows which go through the links should have their total allocation less than link capacity)
  \end{itemize}
\end{itemize}

\subsubsection*{Max-Min Fairness}
\begin{itemize}[topsep=0pt,noitemsep,wide=0pt, leftmargin=\dimexpr\labelwidth + 2\labelsep\relax]
  \item feasible allocation is max-min fair \textit{iff.} increase of any rate within feasible domain \textbf{must be at a cost} of decrease of an already smaller/equal rate
  \item $\boldsymbol{x}$ is max-min fair if for any feasible $\boldsymbol{y}$, if $y_i > x_i$ then $\exists j$ s.t. $y_i < x_j \leq x_i$
  \item \textbf{\underline{always exists}} a max-min fair solution, and \textbf{\underline{always unique}}
\end{itemize}

\subsubsubsection{Max-min fair example}
\begin{multicols*}{2}
\includegraphics*[width=4.1cm, height=1.2cm]{images/max-min.png}
\begin{itemize}[topsep=0pt,noitemsep,wide=0pt, leftmargin=\dimexpr\labelwidth + 2\labelsep\relax]
  \item $C = (3,6,7)$ $d = (2,4,4)$
  \item Resource $C_3$ not bottleneck
  \item $C_2$ Fairshare: $x_1 = x_2 = x_3 = 2$
  \item $C_1$ Fairshare: $x_1 = x_2 = 1.5$
  \item $x_1 = x_2 = 1.5$ 
  \item $x_3 = C_2 - x_1 - x_2 = 3$
\end{itemize}
\end{multicols*}

\subsubsubsection{Bottleneck Resource}
\begin{itemize}[topsep=0pt,noitemsep,wide=0pt, leftmargin=\dimexpr\labelwidth + 2\labelsep\relax]
  \item Assume allocation $\boldsymbol{x}$, in network case, resource $\boldsymbol{r}$ is bottleneck for flow $\boldsymbol{i}$ \textit{iff.}
  \item resource $r$ is \textit{saturated} (fully allocated to all flows)
  \item flow $i$ has the \textit{maximum} rate amongst all flows using resource $r$
  \item flow $i$ CANNOT get more resources from link $r$ if allocation is fair (otherwise hurts flows with lower rates)
\end{itemize}

\framebox{\parbox{\dimexpr\linewidth-1\fboxsep-2\fboxrule}{% 
  $\star$ Theorem: when each flow has $\infty$ demand in a network system, a flow allocation is max-min fair \textit{iff} every flow has a bottleneck resource
}}

\subsubsubsection{Waterfilling Algorithm} 
1. List demands as empty buckets $\rightarrow$ Fill in water until one hits bottleneck
\includegraphics*[width=8cm, height=1.2cm]{images/waterfilling.png}

\subsubsubsection{Weighted Max-Min Fair Share}
\begin{itemize}[topsep=0pt,noitemsep,wide=0pt, leftmargin=\dimexpr\labelwidth + 2\labelsep\relax]
  \item Extension weight vector: $\boldsymbol{\phi} = (\phi_1, \phi_2 \ldots)$
  \begin{itemize}[topsep=0pt,noitemsep,wide=0pt, leftmargin=\dimexpr\labelwidth + 2\labelsep\relax]
    \item no customer receives more than demanded
    \item unsatisfied demand split resource proportional to their weights
  \end{itemize}
  \item $C = 16, \boldsymbol{d} = (4,2,10,4) \phi =(2.5,4,0.5,1)$
  \begin{itemize}[topsep=0pt,noitemsep,wide=0pt, leftmargin=\dimexpr\labelwidth + 2\labelsep\relax]
    \item fair share $x_i = \frac{\phi_i}{\Sigma_i \phi_j} C \Rightarrow \boldsymbol{x} = (5, 8, 1, 2)$
    \item given the upper bound on demand $\Rightarrow x_1 = 4, x_2 = 2$, distribute the rest $10$ proportionally $(3.33, 6.66) \Rightarrow x_4 = 4, x_3 = 6$ 
  \end{itemize}
\end{itemize} 

\begin{multicols*}{2}
  \includegraphics*[width=4.1cm, height=1.2cm]{images/waterfillingweighted.png}
  \begin{itemize}[topsep=0pt,noitemsep,wide=0pt, leftmargin=\dimexpr\labelwidth + 2\labelsep\relax]
    \item buckets has a width of $\phi_i$ and volume of $d_i$
    \item Max-min solution (same example as before) for $\phi = (1,1,1)$ 
  \end{itemize}
\end{multicols*}

\subsection{4. Software Defined Networking}
Implement and manage networks based on new fundemental principles

\subsubsubsection{Networks vs Other Systems}
\begin{itemize}[topsep=0pt,noitemsep,wide=0pt, leftmargin=\dimexpr\labelwidth + 2\labelsep\relax]
  \item computation and storage have been virtualized $\rightarrow$ flexible + managable infra
  \item networks still hard to manage $\rightarrow$ heavily rely on network admins
  \item hard to evolve due to new innovations in systems software $\rightarrow$ OS, languages
  \item routing algorithms change very slowly, network mgmt extremely primitive
\end{itemize}

\subsubsubsection{Network Control Problem}
\begin{itemize}[topsep=0pt,noitemsep,wide=0pt, leftmargin=\dimexpr\labelwidth + 2\labelsep\relax]
  \item Compute the configuration of each phyiscal device
  \item Operate w/o communication gurantees \& network-level protocol (\verb|RIP|, \verb|OSPF|)
\end{itemize}

\subsubsection{Router/Switch at the Core}
\begin{multicols*}{2}
  \includegraphics*[width=4.1cm, height=1.7cm]{images/routingvsforwarding.png}
  \begin{itemize}[topsep=0pt,noitemsep,wide=0pt, leftmargin=\dimexpr\labelwidth + 2\labelsep\relax]
    \item run \textit{routing algorithm} protocols (RIP, OSPF, BGP)
    \item \textit{forwarding datagrams} from incoming to outgoing link
  \end{itemize}
\end{multicols*}

\subsubsubsection{Control Plane vs Data Plane}
\begin{itemize}[topsep=0pt,noitemsep,wide=0pt, leftmargin=\dimexpr\labelwidth + 2\labelsep\relax]
  \item \textbf{Control:} establish router state
  \begin{itemize}[topsep=0pt,noitemsep,wide=0pt, leftmargin=\dimexpr\labelwidth + 2\labelsep\relax]
    \item determine how/where packets are forwarded
    \item routing, traffic engineering, firewall
    \item slow time-scales (per control event)
  \end{itemize}
  \item \textbf{Data:} process/deliver packets
  \begin{itemize}[topsep=0pt,noitemsep,wide=0pt, leftmargin=\dimexpr\labelwidth + 2\labelsep\relax]
    \item based on state in routers and endpoints
    \item e.g. IP, TCP, Ethernet etc
    \item fast timescales (per packet)
  \end{itemize}
  \item \textbf{Control Plane:} Routing Table (RIB) $\rightarrow$ \textbf{Data Plane:} Forwarding Table (FIB)
\end{itemize}

\subsubsubsection{Principle 1: Disaagregation}
\begin{itemize}[topsep=0pt,noitemsep,wide=0pt, leftmargin=\dimexpr\labelwidth + 2\labelsep\relax]
  \item Separation of Control/Data plane $\rightarrow$ should be codified in an open interface \\ 
  \includegraphics*[width=7cm, height=3cm]{images/controldataplane.png}
  \item \textbf{Implications}
  \begin{itemize}[topsep=0pt,noitemsep,wide=0pt, leftmargin=\dimexpr\labelwidth + 2\labelsep\relax]
    \item Network operators able to purchase control/data planes from vendors X/Y
    \item Data plane consists of cheaper commodity forwarding devices (bare-metal switches) BUT Needs to define a \textcolor{red}{forwarding abstraction}
    \begin{itemize}[topsep=0pt,noitemsep,wide=0pt, leftmargin=\dimexpr\labelwidth + 2\labelsep\relax]
      \item general purpose way for control plane to tell data plane to forward pkts in particular way $\rightarrow$ eg. OpenFlow's Flow rules 
      \item Flow rule is \underline{match-action} pair, any packet matched have the \textit{associated action} applied to it
    \end{itemize}
  \end{itemize}
  \item \textbf{Benefits}
  \begin{itemize}[topsep=0pt,noitemsep,wide=0pt, leftmargin=\dimexpr\labelwidth + 2\labelsep\relax]
    \item New market landscape and value shift $\rightarrow$ shift control from vendors to operators that build networks to satisfy users' needs
    \item Opportunities for fast innovations $\rightarrow$ independent evoluation/developmenet, software control of network can evolve independently of hardware
  \end{itemize}
\end{itemize}

\subsubsubsection{Control vs Configuration}
\begin{itemize}[topsep=0pt,noitemsep,wide=0pt, leftmargin=\dimexpr\labelwidth + 2\labelsep\relax]
  \item \textbf{Control:} making real-time decisions about how to respond to link/switch failures. Learn about failure and provide remedy in ms.
  \item \textbf{Configuration:} Operators need ot configure switches and routers, using CLI to update RIB. Interface is capable of installing new routes, which on surface seems equivalent to isntalling new flow rule.
  \begin{enumerate}[topsep=0pt,noitemsep,wide=0pt, leftmargin=\dimexpr\labelwidth + 2\labelsep\relax]
    \item run software that implements control plane \textit{on-switch}
    \begin{itemize}[topsep=0pt,noitemsep,wide=0pt, leftmargin=\dimexpr\labelwidth + 2\labelsep\relax]
      \item implies switches operate autonomously, communicate with peer switches throughout network to construct local routing tables
    \end{itemize}
    \item make control plane physically decoupled from data plane
    \begin{itemize}[topsep=0pt,noitemsep,wide=0pt, leftmargin=\dimexpr\labelwidth + 2\labelsep\relax]
      \item implies control plane implemented \textit{off-switch}, possible to make it logically centralised
    \end{itemize}
  \end{enumerate}
\end{itemize}

\subsubsubsection{Principle 2: Centralised Control}
\begin{itemize}[topsep=0pt,noitemsep,wide=0pt, leftmargin=\dimexpr\labelwidth + 2\labelsep\relax]
  \item Centralised decisions easier to make
  \item From FIB to Forwarding pipeline
  \begin{itemize}[topsep=0pt,noitemsep,wide=0pt, leftmargin=\dimexpr\labelwidth + 2\labelsep\relax]
    \item each table focuses on subset of header fields(may be involved in flow rule)
    \item pkt processed by mulitple tables sequentially, determine how its forwarded
    \includegraphics*[width=8cm, height=1.5cm]{images/openflowswitch.png}
  \end{itemize}
  \item How to implement data plane?
  \begin{itemize}[topsep=0pt,noitemsep,wide=0pt, leftmargin=\dimexpr\labelwidth + 2\labelsep\relax]
    \item fixed-function data plane
    \begin{itemize}[topsep=0pt,noitemsep,wide=0pt, leftmargin=\dimexpr\labelwidth + 2\labelsep\relax]
      \item as header fields are well-known + easy to compute offsets in every pkt
      \item inital idea was purposely data plane agnostic-SDN, focused on opening control plane to \textit{\textbf{programmability}}
    \end{itemize}
    \item programmable data plane
    \begin{itemize}[topsep=0pt,noitemsep,wide=0pt, leftmargin=\dimexpr\labelwidth + 2\labelsep\relax]
      \item performance optimisation, potential changes to protocols
      \item \textbf{Easy network management:} management goals as policies, debug/check behaviour easily
      \item \textbf{Rapid innovation and fast evolution:} enable new services and better performance, detailed configurations are done by the controller
      \item \textbf{Control shift:} vendor $\rightarrow$ operators $\rightarrow$ users
    \end{itemize}
  \end{itemize}
\end{itemize}

\subsubsubsection{Three layers of abstractions}
\includegraphics*[width=8cm, height=3.5cm]{images/sdnabstractions.PNG}
\begin{itemize}[topsep=0pt,noitemsep,wide=0pt, leftmargin=\dimexpr\labelwidth + 2\labelsep\relax]
  \item \textbf{Specification abstraction (northbound API)} $\rightarrow$ Allow control app to express desired network behaviour without implementation
  \item \textbf{Distribution abstraction} (internal to control plane) $\rightarrow$ shield SDN apps from distributed states, making distributed control logically centralized
  \item \textbf{Forwarding abstraction} (southbound to open interface) $\rightarrow$ Allow any forwarding behaviour desired by apps + hiding details of underlying hardware
\end{itemize}

\subsubsubsection{Networking Operating System (NOS)}
\includegraphics*[width=8cm, height=2.5cm]{images/osvsnos.PNG}
OS provides high-level abstractions that make it easier to write applications, an NOS makes it easier to implement control functionality 

\includegraphics*[width=8cm, height=3.3cm]{images/sdnlayers.PNG}


\subsection{5. Use Cases, OpenFlow and ONOS}
Users of SDN:
\begin{enumerate}[topsep=0pt,noitemsep,wide=0pt, leftmargin=\dimexpr\labelwidth + 2\labelsep\relax]
  \item Cloud Providers: Google, Facebook, Microsoft, Opensource Components
  \item Network Operatros: Comcast, AT\&T, NTT
  \item Enterprises: universities/private companies $\rightarrow$ managed edge services/SDN
\end{enumerate}

\subsubsubsection{Case 1: Network Virtulization}
\begin{itemize}[topsep=0pt,noitemsep,wide=0pt, leftmargin=\dimexpr\labelwidth + 2\labelsep\relax]
  \item Exisiting virtualization solutions
  \begin{itemize}[topsep=0pt,noitemsep,wide=0pt, leftmargin=\dimexpr\labelwidth + 2\labelsep\relax]
    \item compute virtualization: VMs, containers | networks: VPNs/VLANs
    \item limted scope: virtualizing the address space
  \end{itemize}
  \item Insight: need for modern cloud: networks to be programmatically created/managed  (\textit{without manual configuration})
  \begin{itemize}[topsep=0pt,noitemsep,wide=0pt, leftmargin=\dimexpr\labelwidth + 2\labelsep\relax]
    \item Disagregation: single API entry point to create/modify/delete VNs
    \item Virtual Networks (VNs) has its own private address spaces
  \end{itemize}
\end{itemize}
\includegraphics*[width=8.5cm, height=2.5cm]{images/networkvirtualization.PNG}

\subsubsubsection{Case 2: Switching Fabrics}
\begin{multicols}{2}
  \begin{itemize}[topsep=0pt,noitemsep,wide=0pt, leftmargin=\dimexpr\labelwidth + 2\labelsep\relax]
    \item \textbf{Cloud Datacenters:} lower costs, newer features
    \item \textbf{Leaf-spine topology:}
    \begin{itemize}[topsep=0pt,noitemsep,wide=0pt, leftmargin=\dimexpr\labelwidth + 2\labelsep\relax]
      \item 2-hop multi-path for rack to rack
      \item 2-hop for intra-rack server-to-server path
      \item 4-hop for inter-rack server-to-server path
    \end{itemize}
  \end{itemize}
  \columnbreak
  \includegraphics*[width=4.2cm, height=2.5cm]{images/switchingfabrics.PNG}
\end{multicols}

\subsubsubsection{Case 3: Traffic Engineering}
\begin{itemize}[topsep=0pt,noitemsep,wide=0pt, leftmargin=\dimexpr\labelwidth + 2\labelsep\relax]
  \item For wide-area links between datacenters
  \includegraphics*[width=8.2cm, height=1.5cm]{images/trafficengineering.PNG}
  \item Traffic classes with priorities $\rightarrow$ delay tolerance vs availability requirements
\end{itemize}

\subsubsubsection{Case 4: Software-Defined WANs}
\includegraphics*[width=8.5cm, height=4.5cm]{images/softwaredefinedwan.PNG}

\subsubsubsection{Use Cases}
\begin{itemize}[topsep=0pt,noitemsep,wide=0pt, leftmargin=\dimexpr\labelwidth + 2\labelsep\relax]
  \item Killer applications: network virtualization, datacenter/cloud computing
  \item More applications like: (1) last-mile access networks, (2) software-defined internet exchange (SDX), (3) mobility and wireless, (4) security, measurement and monitoring
\end{itemize}

\subsubsection{OpenFlow Protocol and Switch}
\includegraphics*[width=8.5cm, height=6.5cm]{images/openflowprotocol.PNG}
\begin{itemize}[topsep=0pt,noitemsep,wide=0pt, leftmargin=\dimexpr\labelwidth + 2\labelsep\relax]
  \item \textbf{OF protocol:} Open southbound API, \textbf{OF Switch:} forwarding abstraction
\end{itemize}
\includegraphics*[width=8.5cm, height=5.5cm]{images/openflowpipeline.PNG}

\subsubsubsection{Terminology for each packet from packet flow}
\begin{itemize}[topsep=0pt,noitemsep,wide=0pt, leftmargin=\dimexpr\labelwidth + 2\labelsep\relax]
  \item \textbf{Header and header field}
  \item \textbf{Pipeline fields:} values attached to packet during pipeline processing $\rightarrow$ e.g. ingress port and metadata
  \item \textbf{Action:} operation that acts on packet: drop/forward to port/modify TTL
  \item \textbf{Action set:} acumulated while processed by flow tables, executed at the end of the pipeline processing
\end{itemize}


\subsubsubsection{OpenFlow flow entry}
\begin{itemize}[topsep=0pt,noitemsep,wide=0pt, leftmargin=\dimexpr\labelwidth + 2\labelsep\relax]
  \item Flow entry in flow table looks like:
  \begin{itemize}[topsep=0pt,noitemsep,wide=0pt, leftmargin=\dimexpr\labelwidth + 2\labelsep\relax]
    \item \texttt{Match Fields $|$ Priority $|$ Counters $|$ Instructions $|$ Timeouts}
  \end{itemize}
  \item Match field: packets are matched against
  \begin{itemize}[topsep=0pt,noitemsep,wide=0pt, leftmargin=\dimexpr\labelwidth + 2\labelsep\relax]
    \item Header fields and pipeline fields, may be wildcarded (any) or bitmasked (subset of bits)
  \end{itemize}
  \item Priority: used to choose from multiple matches
  \item Instruction Set
  \begin{itemize}[topsep=0pt,noitemsep,wide=0pt, leftmargin=\dimexpr\labelwidth + 2\labelsep\relax]
    \item contains list of actions to apply immediately, and a set of actions to add to the action set
    \item Modify pipeline processing (go to another flow table)
  \end{itemize}
  \item Default entry: table-miss flow entry $\rightarrow$ send packet to control plane OR drop the packet
\end{itemize}

\includegraphics*[width=8.5cm, height=9.5cm]{images/ingressegress.PNG}


\subsubsubsection{Matching and instruction execution}
\includegraphics*[width=8.5cm, height=4.3cm]{images/matchinginstruction.PNG}

\subsubsubsection{Operation of SDN (controller-switch)}
Reactive paradigm: relying on dataplane $\rightarrow$ control plane
\includegraphics*[width=8.5cm, height=4.5cm]{images/operationofsdn.PNG}

\subsubsubsection{Interfaces of an SDN}
\includegraphics*[width=8.5cm, height=4cm]{images/interfacesofsdn.PNG}

\subsubsubsection{ONOS Control Plane}
\begin{itemize}[topsep=0pt,noitemsep,wide=0pt, leftmargin=\dimexpr\labelwidth + 2\labelsep\relax]
  \item ONOS model: Open Northbound API, specification abstraction
  \item ONOS system: Distribution abstraction
  \item ONOS abstractions:
  \begin{itemize}[topsep=0pt,noitemsep,wide=0pt, leftmargin=\dimexpr\labelwidth + 2\labelsep\relax]
    \item \textbf{Intent:} high-level intents, network-wide, topology-independent programming constructs $\rightarrow$ e.g. intent to custom IP reach target IP
    \item \textbf{Flow Objective:} finer-grained control, device-centric, programming constructs and also \textit{pipeline-independent}
    \item \textbf{Flow Rule:} use to control various pipelines, fix-function/programmable
  \end{itemize}
\end{itemize}

\subsubsubsection{Network Operating System (NOS)}
\begin{itemize}[topsep=0pt,noitemsep,wide=0pt, leftmargin=\dimexpr\labelwidth + 2\labelsep\relax]
  \item Like any other horizontally scalable cloud application
  \item Consists of a set of loosely coupled subsytems
  \begin{itemize}[topsep=0pt,noitemsep,wide=0pt, leftmargin=\dimexpr\labelwidth + 2\labelsep\relax]
    \item Each for an aspect e.g. topology, host tracking
    \item maintains its own service abstraction
  \end{itemize}
  \item often associated with a micro-service architecture
  \item includes a scalable and highly available key/value store
\end{itemize}

\subsubsubsection{Architecture of ONOS}
\includegraphics*[width=8.5cm, height=4.5cm]{images/opennetworkoperatingsystem.PNG}
\begin{itemize}[topsep=0pt,noitemsep,wide=0pt, leftmargin=\dimexpr\labelwidth + 2\labelsep\relax]
  \item \textbf{Northbound Interfaces(NBI)}
  \begin{itemize}[topsep=0pt,noitemsep,wide=0pt, leftmargin=\dimexpr\labelwidth + 2\labelsep\relax]
    \item apps use to stay informed about \textit{network state} (topology, intercept packets) AND to \textit{control network} data plane
  \end{itemize}
  \item \textbf{Distributed Core}
  \begin{itemize}[topsep=0pt,noitemsep,wide=0pt, leftmargin=\dimexpr\labelwidth + 2\labelsep\relax]
    \item responsible for managing network state, notifying apps about state changes
    \item internal: scalable key/value store (Atomix)
  \end{itemize}
  \item \textbf{Southbound Interface (SBI)}: constructed from a set of plugins including shared protocol libs/device-specific drivers
\end{itemize}

\subsection{6. P4 SDN}
\subsubsubsection{Issues with OpenFlow}
\begin{itemize}[topsep=0pt,noitemsep,wide=0pt, leftmargin=\dimexpr\labelwidth + 2\labelsep\relax]
  \item Data-plane protocol evolution requires frequent changes to standards ($12 \rightarrow 40$ OpenFlow match fields now)
  \item Limited interoperability between vendors (OpenFlow / netconf / JSON / XML variants) \& Limited programmability
\end{itemize}

\subsubsubsection{Software Stack of an SDN + Bare-metal switches}
Bottom is a bare-metal switch, with Switch OS on top of the Merchant Silicon
\begin{itemize}[topsep=0pt,noitemsep,wide=0pt, leftmargin=\dimexpr\labelwidth + 2\labelsep\relax]
  \item Each packet is processed by the NPU through:
  \begin{itemize}[topsep=0pt,noitemsep,wide=0pt, leftmargin=\dimexpr\labelwidth + 2\labelsep\relax]
    \item Multi-stage Forwarding pipeline, stages are fixed-function/programmable
  \end{itemize}
\end{itemize}

\includegraphics*[width=8.5cm, height=4cm]{images/baremetalswitch.PNG}
\includegraphics*[width=8.5cm, height=4cm]{images/softwarestackofsdn.PNG}

\subsubsection{P4}
\includegraphics*[width=8.5cm, height=3.1cm]{images/p4.PNG}
\begin{itemize}[topsep=0pt,noitemsep,wide=0pt, leftmargin=\dimexpr\labelwidth + 2\labelsep\relax]
  \item \textbf{Domain-specific language} to formally defined dataplane pipeline
  \begin{itemize}[topsep=0pt,noitemsep,wide=0pt, leftmargin=\dimexpr\labelwidth + 2\labelsep\relax]
    \item Describe protocol headers, lookup tables actions, counters 
    \item Can describe fast (e.g. ASIC, FPGA) / slow pipelines (SW switch)
  \end{itemize}
  \item \textbf{Good for programmable switches/fixed-function ones}
  \begin{itemize}[topsep=0pt,noitemsep,wide=0pt, leftmargin=\dimexpr\labelwidth + 2\labelsep\relax]
    \item Defines `contract' between control plane/dataplane for runtime control
  \end{itemize}
\end{itemize}

\subsubsubsection{Protocol Independent Switch Architecture}
\begin{itemize}[topsep=0pt,noitemsep,wide=0pt, leftmargin=\dimexpr\labelwidth + 2\labelsep\relax]
  \item \textbf{Parser:} declare headers that should be recognized + their order in the packet
  \item \textbf{Match-Action Pipeline:} define tables and the exact processing algorithm
  \item \textbf{Deparser:} declare how output packet will look on the wire
  \item Process:
  \begin{enumerate}[topsep=0pt,noitemsep,wide=0pt, leftmargin=\dimexpr\labelwidth + 2\labelsep\relax]
    \item Packet parsed into individual headers (parsed representation)
    \item Headers and intermediate results can be used for matching/actions, can be modified, added or removed
    \item Packet is deparsed (serialized) after the pipeline
  \end{enumerate}
\end{itemize}
\includegraphics*[width=8.5cm, height=4.5cm]{images/pisalogicalpipeline.PNG}

\subsubsubsection{From logical to physical}
\begin{itemize}[topsep=0pt,noitemsep,wide=0pt, leftmargin=\dimexpr\labelwidth + 2\labelsep\relax]
  \item P4 needs to account for different chips with different physical pipelines
  \item If only one logical pipeline, P4 compiler's job is easy, but marketplace not converged, many vendors for high-speed NPUs
  \item Allow vendor specific logical pipeline, through different architectures (\verb|arch.p4| provided by vendors) $P4_{14} \rightarrow P4_{16}$ 
\end{itemize}

\includegraphics*[width=8.5cm, height=4.5cm]{images/p416languageelements.PNG}

\subsubsection{P4\_16 approach}
\includegraphics*[width=8.5cm, height=4cm]{images/programmingp4target.PNG}
\begin{itemize}[topsep=0pt,noitemsep,wide=0pt, leftmargin=\dimexpr\labelwidth + 2\labelsep\relax]
  \item \textbf{P4 Target:} An embodiment of a specific hardware implementation
  \begin{itemize}[topsep=0pt,noitemsep,wide=0pt, leftmargin=\dimexpr\labelwidth + 2\labelsep\relax]
    \item specific device/platformn (sw/hw), refers to actual entity processing network packets according to P4 code
    \item \textit{purpose:} provides concrete execution env where P4 program deployed
    \item \textit{characteristics:} has own hw capabilities, vendor-specific optimization/constraints affect how P4 code is mapped/executed
  \end{itemize}
  \item \textbf{P4 Architecture}: Provides an interface to program a target via some set of P4-programmable components, externs, fixed components
  \begin{itemize}[topsep=0pt,noitemsep,wide=0pt, leftmargin=\dimexpr\labelwidth + 2\labelsep\relax]
    \item high-level abstraction define how packets flow through programmable pipeline, describes logical pipeline as \underline{general strucure and processing behaviour of P4 programs}
  \end{itemize}
\end{itemize}

\subsubsubsection{Behavioral Model (bmv2)}
\begin{itemize}[topsep=0pt,noitemsep,wide=0pt, leftmargin=\dimexpr\labelwidth + 2\labelsep\relax]
  \item \verb|bmv2| allows developers implement their own P4-programmable architecture as a software switch
  \item think of \verb|bmv2| as the hardware of the switch, \verb|psa_switch| target supports PSA architecture, \verb|simple_switch| supports V1Model
  \item \textbf{simple\_router.p4}
  \begin{multicols}{2}
    \begin{itemize}[topsep=0pt,noitemsep,wide=0pt, leftmargin=\dimexpr\labelwidth + 2\labelsep\relax]
      \item \textit{Data plane program (P4):} Defines match-action tables, performs lookup and executes chosen runtime
      \item \textit{Control plane (runtime):} Populates table entries with specific information, based on config, automatic discovery and protocol calculations
    \end{itemize}
    \columnbreak
    \includegraphics*[width=4cm, height=3cm]{images/simple_router.PNG}
  \end{multicols}
\end{itemize}

\includegraphics*[width=8.5cm, height=4.5cm]{images/tablematchaction.PNG}

\subsection{7. Internet Interconnection}
\subsubsection{Autonomous System}
\begin{itemize}[topsep=0pt,noitemsep,wide=0pt, leftmargin=\dimexpr\labelwidth + 2\labelsep\relax]
  \item Network of interconnected routers, identified by unique AS number (ASN)
  \item Control by a single administrative domain (company can have multiple ASN)
  \item Use common routing protocol and policies
  \item \textit{How to obtain AS number?}
  \begin{itemize}[topsep=0pt,noitemsep,wide=0pt, leftmargin=\dimexpr\labelwidth + 2\labelsep\relax]
    \item IANA: Internet Assigned Numbers Authority $\rightarrow$ consists of 5 regional internet registries OR AS Databases
  \end{itemize}
\end{itemize}

\subsubsubsection{Internet structure: Network of Networks}
\includegraphics*[width=8.5cm, height=3.5cm]{images/internetstructure.PNG}
\begin{itemize}[topsep=0pt,noitemsep,wide=0pt, leftmargin=\dimexpr\labelwidth + 2\labelsep\relax]
  \item at center: small \# of well-connected large networks
  \begin{itemize}[topsep=0pt,noitemsep,wide=0pt, leftmargin=\dimexpr\labelwidth + 2\labelsep\relax]
    \item \textcolor{red}{tier-1 commercial ISPs:} (e.g., Level 3, Sprint, AT\&T, NTT), national \& international coverage
    \item \textcolor{red}{content provider:} (e.g., Google): private network that connects it data centers to Internet, often bypassing tier-1, regional ISPs
  \end{itemize}
\end{itemize}

\includegraphics*[width=8.5cm, height=4.5cm]{images/asgraphs.PNG}

\subsubsection{Internet Peering | How are they connected?}
\begin{itemize}[topsep=0pt,noitemsep,wide=0pt, leftmargin=\dimexpr\labelwidth + 2\labelsep\relax]
  \item \textit{peering:} voluntary interconnection of separate internet networks: purpose of exchanging traffic between `down-stream' users of each network
  \item bilateral agreement between neighbor ASes
  \item depends on business relationships: (1) customer-provider r/s or (2) peer-to-peer (settlement-free peering) r/s
\end{itemize}

\subsubsubsection{Customer and Providers}
\begin{multicols}{2}
  \includegraphics*[width=4cm, height=2cm]{images/customerprovider.png}
  \columnbreak
  \begin{itemize}[topsep=0pt,noitemsep,wide=0pt, leftmargin=\dimexpr\labelwidth + 2\labelsep\relax]
    \item Customer pays provider for access to internet, reachable from anyone
    \item Provider provides \underline{\textit{transit service}} for the customer
  \end{itemize}
\end{multicols}

\subsubsubsection{Nontransit vs Transit ASes}
\begin{multicols}{2}
  \includegraphics*[width=4cm, height=2cm]{images/nontransitvstransit.png}
  \columnbreak
  \begin{itemize}[topsep=0pt,noitemsep,wide=0pt, leftmargin=\dimexpr\labelwidth + 2\labelsep\relax]
    \item customer does not allow traffic to go through it
    \item NET A has 2 providers: \textbf{multi-honing}
    \item traffic should NEVER flow from P1 through NET A to P2
    \item non-transit AS might be a corporate/campus network/`content provider'
  \end{itemize}
\end{multicols}

\subsubsubsection{Selective Transit}
\begin{multicols}{2}
  \includegraphics*[width=4cm, height=2cm]{images/selectivetransit.png}
  \columnbreak
  \begin{itemize}[topsep=0pt,noitemsep,wide=0pt, leftmargin=\dimexpr\labelwidth + 2\labelsep\relax]
    \item NET A provides transit between B \& C and C \& D
    \item Net A \underline{DOES NOT} provide transit between D \& B
    \item Most transit networks transit in a selective manner
  \end{itemize}
\end{multicols}

\subsubsubsection{Customers do not always need AS\#}
\begin{multicols}{2}
\includegraphics*[width=4cm, height=2cm]{images/customeras.png}
\columnbreak
\begin{itemize}[topsep=0pt,noitemsep,wide=0pt, leftmargin=\dimexpr\labelwidth + 2\labelsep\relax]
  \item Static routing is the most common way of connecting autonomous routing domain to the internet.
  \item If not helping other networks exchange traffic $\rightarrow$ not needed to have any AS\#
\end{itemize}
\end{multicols}

\subsubsubsection{Customer-Provider Hiearachy}
\includegraphics*[width=8.5cm, height=4cm]{images/customerhiearchy.png}

\subsubsubsection{Peer-to-peer Relationship}
\includegraphics*[width=8.5cm, height=4cm]{images/peertopeer.png}

\subsubsubsection{Peering Provides Shortcuts}
\includegraphics*[width=8.5cm, height=3.5cm]{images/peeringshortcuts.png}
\begin{itemize}[topsep=0pt,noitemsep,wide=0pt, leftmargin=\dimexpr\labelwidth + 2\labelsep\relax]
  \item At top there are Tier 1 ASP
  \begin{itemize}[topsep=0pt,noitemsep,wide=0pt, leftmargin=\dimexpr\labelwidth + 2\labelsep\relax]
    \item These AS does not have any upstream providers, do not pay other ASP for transit services $\rightarrow$ any pair of them must have a peering connection
  \end{itemize}
\end{itemize}

\subsubsubsection{Peering Dilemma}
\begin{itemize}[topsep=0pt,noitemsep,wide=0pt, leftmargin=\dimexpr\labelwidth + 2\labelsep\relax]
  \item \textbf{To Peer}
  \begin{itemize}[topsep=0pt,noitemsep,wide=0pt, leftmargin=\dimexpr\labelwidth + 2\labelsep\relax]
    \item Reduce upstream transit costs, improve end-to-end performance
    \item Be the only way to connect sutomers to some part of internet (tier-1)
  \end{itemize}
  \item \textbf{Not Peer}
  \begin{itemize}[topsep=0pt,noitemsep,wide=0pt, leftmargin=\dimexpr\labelwidth + 2\labelsep\relax]
    \item Rather have customers than peers, usually peers are competition
    \item Peering relationships may require periodic renegotiation
  \end{itemize}
\end{itemize}

\subsubsubsection{Tier 1 ASes/ISPs}
\begin{itemize}[topsep=0pt,noitemsep,wide=0pt, leftmargin=\dimexpr\labelwidth + 2\labelsep\relax]
  \item Have access to entire Internet through settlement-free peering links
  \item Top of customer-provider hierarchy, typically large (inter)national backbones
  \item Have no upstream provider, peer with one another to form a full-mesh ($\approx 10-12$ ASes $\rightarrow$ AT\&T, Sprint, Level3)
\end{itemize}

\subsubsubsection{Other ASes}
\begin{itemize}[topsep=0pt,noitemsep,wide=0pt, leftmargin=\dimexpr\labelwidth + 2\labelsep\relax]
  \item Lower layer providers (tier-2)
  \begin{itemize}[topsep=0pt,noitemsep,wide=0pt, leftmargin=\dimexpr\labelwidth + 2\labelsep\relax]
    \item provide transit to downstream customers, need $\geq 1$ provider of their own
    \item typically have national or regional scope $\rightarrow$ include $1000+$ ASes
  \end{itemize}
  \item Stub ASes:
  \begin{itemize}[topsep=0pt,noitemsep,wide=0pt, leftmargin=\dimexpr\labelwidth + 2\labelsep\relax]
    \item Do not provide transit service, but connect to upstream provider(s)
    \item Most ASes are like these ($\approx 85-90\%$) e.g. NUS
  \end{itemize}
\end{itemize}

\subsubsubsection{Valley-free Property}
\begin{multicols}{2}
  \begin{itemize}[topsep=0pt,noitemsep,wide=0pt, leftmargin=\dimexpr\labelwidth + 2\labelsep\relax]
    \item \textbf{Valid AS paths (from routing tables)}
    \begin{itemize}[topsep=0pt,noitemsep,wide=0pt, leftmargin=\dimexpr\labelwidth + 2\labelsep\relax]
      \item single peak (uphill + downhill)
      \item single flat top (uphill + 1 peering + downhill)
      \item any sub-paths of above are valid
    \end{itemize}
    \item \textbf{Invalid patterns}
    \begin{itemize}[topsep=0pt,noitemsep,wide=0pt, leftmargin=\dimexpr\labelwidth + 2\labelsep\relax]
      \item provider $\rightarrow$ customer $\rightarrow$ peering
      \item provider $\rightarrow$ customer $\rightarrow$ provider
      \item peering $\rightarrow$ peering
      \item peering $\rightarrow$ provider
    \end{itemize}
  \end{itemize}
\end{multicols}

\subsection{8. Inter-Domain Routing and Policy}
\subsubsubsection{Challenges for Inter-domain Routing}
\begin{itemize}[topsep=0pt,noitemsep,wide=0pt, leftmargin=\dimexpr\labelwidth + 2\labelsep\relax]
  \item \textbf{Scale}
  \begin{itemize}[topsep=0pt,noitemsep,wide=0pt, leftmargin=\dimexpr\labelwidth + 2\labelsep\relax]
    \item millions of routers, $200,000+$ prefixes
    \item $35K+$ self-operated networks, $50K+$ ASes
  \end{itemize}
  \item \textbf{Privacy:} ASes don't want to expose internal topologies/business relationships with neighbours
  \item \textbf{Policy}
  \begin{itemize}[topsep=0pt,noitemsep,wide=0pt, leftmargin=\dimexpr\labelwidth + 2\labelsep\relax]
    \item No Internet-wide notion of a link cost metricc
    \item Each AS needs to control over where it sends traffic and who can send traffic through it
  \end{itemize}
\end{itemize}

\subsubsubsection{Routing Algorithms}
\begin{itemize}[topsep=0pt,noitemsep,wide=0pt, leftmargin=\dimexpr\labelwidth + 2\labelsep\relax]
  \item \textbf{Link state algorithm:} all routers have complete topology, link cost info
  \begin{itemize}[topsep=0pt,noitemsep,wide=0pt, leftmargin=\dimexpr\labelwidth + 2\labelsep\relax]
    \item Global/centralized: Dijsktra's | Open Shortest Path First (OSPF)
    \item Limitation:
    \begin{itemize}[topsep=0pt,noitemsep,wide=0pt, leftmargin=\dimexpr\labelwidth + 2\labelsep\relax]
      \item Topology information flooded $\rightarrow$ high bandwidth/storage overhead, nodes divulge sensitive information
      \item Entire path computed locally per node $\rightarrow$ high processing overhead
      \item Minimize notion of total distance $\rightarrow$ only if policy shared/uniform
    \end{itemize}
  \end{itemize}
  \item \textbf{Distance vector algorithm:} router knows connected neighbours' link costs
  \begin{itemize}[topsep=0pt,noitemsep,wide=0pt, leftmargin=\dimexpr\labelwidth + 2\labelsep\relax]
    \item Decentralized algorithm: Bellman-Ford, iterative process of computation, exchange info with neighbours | Routing Information Protocol (RIP)
    \item Advantages: hide details of network topology, next hop determined/node
    \item Disadvantages: minimize some notion of total distance, slow convergence
  \end{itemize}
\end{itemize}

\subsubsection*{Path-Vector Routing}
\includegraphics*[width=8.5cm, height=1.5cm]{images/pathvectorrouting.png}
\begin{itemize}[topsep=0pt,noitemsep,wide=0pt, leftmargin=\dimexpr\labelwidth + 2\labelsep\relax]
  \item Extension of distance-vector routing, support flexible routing policies
  \item advertise entire path, \textit{DV} send distance metric per destination $d$
  \item \textit{PV} send entire path for each destination $d$
  \item Faster loop detection:
  \begin{itemize}[topsep=0pt,noitemsep,wide=0pt, leftmargin=\dimexpr\labelwidth + 2\labelsep\relax]
    \item Node can easily detect loops, check if itself is in path
    \item Node can then discard paths with loops
  \end{itemize}
\end{itemize}

\subsubsection{Border Gateway Protocol (BGP)}
\includegraphics*[width=8.5cm, height=4.4cm]{images/bgpoperations.png}
\begin{itemize}[topsep=0pt,noitemsep,wide=0pt, leftmargin=\dimexpr\labelwidth + 2\labelsep\relax]
  \item Allows subnet to advertise its existence to rest of internet: \verb|`I am here'|
  \item Allows ASes to determine good routes from other networks based on reachability info/policy
\end{itemize}

\subsubsubsection{BGP/IGP model used in ISPs}
\begin{itemize}[topsep=0pt,noitemsep,wide=0pt, leftmargin=\dimexpr\labelwidth + 2\labelsep\relax]
  \item \textbf{eBGP:} exchange reachability info from neighbor ASes
  \includegraphics*[width=8cm, height=2cm]{images/ebgp.png}
  \begin{itemize}[topsep=0pt,noitemsep,wide=0pt, leftmargin=\dimexpr\labelwidth + 2\labelsep\relax]
    \item exterior BGP peering (eBGP), between BGP speakers in different ASes $\rightarrow$ should be directly connected, no intermediate hop
    \item AS3 advertises an IP prefix to AS1:
    \begin{itemize}[topsep=0pt,noitemsep,wide=0pt, leftmargin=\dimexpr\labelwidth + 2\labelsep\relax]
      \item AS3 promises it will forward packets towards that prefix
      \item AS3 can aggregate prefixes in its advertisement
    \end{itemize}
  \end{itemize}
  \item \textbf{iBGP:} propagate reachability info across backbone; carry ISP’s own customer prefixes
  \includegraphics*[width=8cm, height=2cm]{images/ibgp.png}
  \begin{itemize}[topsep=0pt,noitemsep,wide=0pt, leftmargin=\dimexpr\labelwidth + 2\labelsep\relax]
    \item peers within an AS; not required to be directly connected $\Rightarrow$ IGP (RIP or OSPF) takes care of inter-BGP speaker connectivity
    \item iBGP peers must be (logically) fully meshed
    \begin{itemize}[topsep=0pt,noitemsep,wide=0pt, leftmargin=\dimexpr\labelwidth + 2\labelsep\relax]
      \item used to originate connected local networks
      \item used to pass on prefixes learned from outside the AS
      \item do not pass on prefixes learned form other iBGP speakers
    \end{itemize}
    \item 1c can use iBGP to distribute prefix info to all routers in AS1; 1b can re-advertise info to AS2 over eBGP
  \end{itemize}
\end{itemize}

\subsubsubsection{BGP Messages}
\begin{itemize}[topsep=0pt,noitemsep,wide=0pt, leftmargin=\dimexpr\labelwidth + 2\labelsep\relax]
  \item \verb|OPEN|: opens TCP connection to peer and authenticates sender
  \item \verb|UPDATE|: advertises new paths (or withdraws old paths)
  \item \verb|KEEPALIVE|: keeps conn. alive in absence of \verb|UPDATES|/ACKs OPEN req
  \item \verb|NOTIFICATION|: reports err. in previous mssages/used to close conn
\end{itemize}

\subsubsubsection{UPDATE Message Format}
\includegraphics*[width=8cm, height=1.5cm]{images/updatemessageformat.png}
\begin{itemize}[topsep=0pt,noitemsep,wide=0pt, leftmargin=\dimexpr\labelwidth + 2\labelsep\relax]
  \item Withdrawn Routes: IP prefixes for the routes withdrawn
  \begin{itemize}[topsep=0pt,noitemsep,wide=0pt, leftmargin=\dimexpr\labelwidth + 2\labelsep\relax]
    \item Can withdraw multiple routes in an UPDATE message
    \item Network Layer Reachability Information (NLRI): IP
  \end{itemize}
  \item prefixes that can be reached from the advertised route
  \begin{itemize}[topsep=0pt,noitemsep,wide=0pt, leftmargin=\dimexpr\labelwidth + 2\labelsep\relax]
    \item Can only advertise one feasible route for the NLRI
    \item IP prefixes are coded more compactly (refer to RFC)
  \end{itemize}
\end{itemize}

\subsubsubsection{Withdrawn Routes}
\begin{itemize}[topsep=0pt,noitemsep,wide=0pt, leftmargin=\dimexpr\labelwidth + 2\labelsep\relax]
  \item No expiration timer for the routes like RIP
  \item Invalid routes are actively withdrawn by the original advertiser
  \item Or use UPDATE message to replace the existing routes
  \item All routes from a peer become invalid when the peer goes down
\end{itemize}

\subsubsubsection{BGP Path Attributes}
\begin{enumerate}[topsep=0pt,noitemsep,wide=0pt, leftmargin=\dimexpr\labelwidth + 2\labelsep\relax]
  \item well-known mandatory: every single route update MUST HAVE
  \begin{itemize}[topsep=0pt,noitemsep,wide=0pt, leftmargin=\dimexpr\labelwidth + 2\labelsep\relax]
    \item \verb|ORIGIN|: conveys the origin of the prefix, learned from
    \begin{itemize}[topsep=0pt,noitemsep,wide=0pt, leftmargin=\dimexpr\labelwidth + 2\labelsep\relax]
      \item i (IGP): an interior gateway protocol such as RIP/OSPF
      \item ? (INCOMPLETE): unknown source
      \item e (EGP): an exterior gateway protocol $\rightarrow$ BGP (used to be EGP)
    \end{itemize}
    \item \verb|AS_PATH|: 
    \begin{itemize}[topsep=0pt,noitemsep,wide=0pt, leftmargin=\dimexpr\labelwidth + 2\labelsep\relax]
      \item contains ASes through which NLRI has passed
      \item expressed as a sequence e.g. AS 79, AS 11 or a set
    \end{itemize}
    \item \verb|NEXT_HOP|: indicates IP address of the router in the next-hop AS
  \end{itemize}
  \item well-known discretionary
  \begin{itemize}[topsep=0pt,noitemsep,wide=0pt, leftmargin=\dimexpr\labelwidth + 2\labelsep\relax]
    \item \verb|LOCAL_PREF|, \verb|ATOMIC_AGGREGATE|
  \end{itemize}
  \item optional transitive: have to be propagated along routing path
  \begin{itemize}[topsep=0pt,noitemsep,wide=0pt, leftmargin=\dimexpr\labelwidth + 2\labelsep\relax]
    \item \verb|AGGREGATOR|, \verb|COMMUNITY|
  \end{itemize}
  \item optional non-transitive
  \begin{itemize}[topsep=0pt,noitemsep,wide=0pt, leftmargin=\dimexpr\labelwidth + 2\labelsep\relax]
    \item \verb|MULTI_EXIT_DISC (MED)|
  \end{itemize}
  \item Some implementation rules:
  \begin{itemize}[topsep=0pt,noitemsep,wide=0pt, leftmargin=\dimexpr\labelwidth + 2\labelsep\relax]
    \item must recognize all well-known attributes
    \item mandatory attributes must be included in \verb|UPDATE| msg that contain NLRI
    \item once a BGP peer updates well-known attributes, it must pass to its peers
  \end{itemize}
\end{enumerate}

\subsubsubsection{How do entries get in forwarding table}
Ties together hierarchical routing with BGP/OSPF, provides overview of BGP
\begin{itemize}[topsep=0pt,noitemsep,wide=0pt, leftmargin=\dimexpr\labelwidth + 2\labelsep\relax]
  \item Router becomes aware of IP prefix
  \includegraphics*[width=8.2cm, height=2.4cm]{images/prefix1.png}
  \begin{itemize}[topsep=0pt,noitemsep,wide=0pt, leftmargin=\dimexpr\labelwidth + 2\labelsep\relax]
    \item BGP message contains “routes”
    \item route = prefix + attributes: AS-PATH, NEXT-HOP,… Example: route:
    \begin{itemize}[topsep=0pt,noitemsep,wide=0pt, leftmargin=\dimexpr\labelwidth + 2\labelsep\relax]
      \item \verb|Prefix|: 138.16.64.29/22; \verb|AS-PATH|: AS3 AS131;
      \item \verb|NEXT-HOP|: 201.44.13.125
    \end{itemize}
    \includegraphics*[width=8.2cm, height=2.4cm]{images/prefix2.png}
    \item Router may receive multiple routes for the \underline{same} destination prefix
    \item Router has to select \textbf{one} route
  \end{itemize}
  \item Router determines output port for IP prefix
  \begin{itemize}[topsep=0pt,noitemsep,wide=0pt, leftmargin=\dimexpr\labelwidth + 2\labelsep\relax]
    \item Select best BGP route to prefix: based on \underline{shortest AS-PATH}
    \item Use selected route's \verb|NEXT-HOP| attribute
    \includegraphics*[width=8.2cm, height=2.4cm]{images/prefix3.png}
    \begin{itemize}[topsep=0pt,noitemsep,wide=0pt, leftmargin=\dimexpr\labelwidth + 2\labelsep\relax]
      \item Route's \verb|NEXT-HOP| attribute is the IP address of router interface that begins the \verb|AS PATH|
      \item e.g. \verb|AS-PATH|: AS2 $\cdots$ AS98; \verb|NEXT-HOP|: 111.99.86.55
      \item Router uses OSPF to find shortest path from 1c to 111.99.86.55
    \end{itemize}
  \end{itemize}
  \item Router enteres the prefix-port in forwarding table
  \includegraphics*[width=8.2cm, height=2.4cm]{images/prefix4.png}
  \begin{itemize}[topsep=0pt,noitemsep,wide=0pt, leftmargin=\dimexpr\labelwidth + 2\labelsep\relax]
    \item Identifies port along the OSPF shortest path
    \item Adds prefix-port entry to its forwarding table $\rightarrow$ (138.16.64/22, port 4)
  \end{itemize}
\end{itemize}

\subsubsubsection{Hot Potato Routing}
\includegraphics*[width=8.2cm, height=2.4cm]{images/prefix5.png}
\begin{itemize}[topsep=0pt,noitemsep,wide=0pt, leftmargin=\dimexpr\labelwidth + 2\labelsep\relax]
  \item If exists multiple best inter-domain routes
  \item Then choose route with closest NEXT-HOP
  \begin{itemize}[topsep=0pt,noitemsep,wide=0pt, leftmargin=\dimexpr\labelwidth + 2\labelsep\relax]
    \item use OSPF to determine which gateway is closest (as interdomain routing highest cost)
    \item Q: From 1c, chose AS3 $\cdots$ AS98 or AS2 $\cdots$ AS98?
    \item A: route AS3 $\cdots$ AS98 since it is closer 
  \end{itemize}
\end{itemize}

\subsubsubsection{BGP Routing Information Bases}
\begin{itemize}[topsep=0pt,noitemsep,wide=0pt, leftmargin=\dimexpr\labelwidth + 2\labelsep\relax]
  \item Route = prefix + attributes = NLRI + path attributes
  \item All routes in a BGP speaker:
  \begin{itemize}[topsep=0pt,noitemsep,wide=0pt, leftmargin=\dimexpr\labelwidth + 2\labelsep\relax]
    \item Routing information Bases (RIBs)
    \item RIBs = Adj-RIBs-In + Loc-RIB + Adj-RIBs-Out
    \item \textbf{Adj-RIBs-In:} unprocessed routes from peers via inbound UPDATE; input for decision making
    \item \textbf{Loc-RIB:} selected local routes used by the router
    \item \textbf{Adj-RIBs-Out:} selected for advertisement to peers
  \end{itemize}
\end{itemize}
\includegraphics*[width=8.2cm, height=4.2cm]{images/bgpdecisionprocessoverview.png}

\subsubsubsection{BGP applying policy to routes}
\begin{itemize}[topsep=0pt,noitemsep,wide=0pt, leftmargin=\dimexpr\labelwidth + 2\labelsep\relax]
  \item \textbf{Import policy:}
  \begin{itemize}[topsep=0pt,noitemsep,wide=0pt, leftmargin=\dimexpr\labelwidth + 2\labelsep\relax]
    \item filter unwanted routes from neighbor $\rightarrow$ prefix your customer doesnt own
    \item used to rank customer routes over peer routes
    \item manipulate attributes to influence path selection $\rightarrow$ assign local preference to favored routes
  \end{itemize}
  \item \textbf{Export policy:}
  \begin{itemize}[topsep=0pt,noitemsep,wide=0pt, leftmargin=\dimexpr\labelwidth + 2\labelsep\relax]
    \item filter routes you don’t want to tell your neighbor $\rightarrow$ export only customer routes to peers \& providers
    \item manipulate attribute to control what they see $\rightarrow$ make paths look artificially longer (AS prepending)
  \end{itemize}
\end{itemize}

\subsubsubsection{BGP how is policy used}
\begin{itemize}[topsep=0pt,noitemsep,wide=0pt, leftmargin=\dimexpr\labelwidth + 2\labelsep\relax]
  \item Objectives: used by commercial ISPs to
  \begin{itemize}[topsep=0pt,noitemsep,wide=0pt, leftmargin=\dimexpr\labelwidth + 2\labelsep\relax]
    \item fulfill bilateral agreements with other ISPs, minimize monetary costs (or maximize revenue) AND ensure good performance for customers
  \end{itemize}
  \item Bilateral agreement between neighbor ISPs
  \begin{itemize}[topsep=0pt,noitemsep,wide=0pt, leftmargin=\dimexpr\labelwidth + 2\labelsep\relax]
    \item defines who will provide transit for what
    \item depends on business relationships: Customer-provider/Peer-to-Peer r/s
  \end{itemize}
\end{itemize}

\subsubsubsection{Business Relationships}
\begin{itemize}[topsep=0pt,noitemsep,wide=0pt, leftmargin=\dimexpr\labelwidth + 2\labelsep\relax]
  \item \textbf{Customer-Provider:} Customer pays provider for access to Internet
  \includegraphics*[width=8.2cm, height=3.2cm]{images/customerprovider.PNG}
  \begin{itemize}[topsep=0pt,noitemsep,wide=0pt, leftmargin=\dimexpr\labelwidth + 2\labelsep\relax]
    \item provider exports customer’s routes to everybody
    \item customer exports provider’s routes to customers
  \end{itemize}
  \item \textbf{Peer-to-Peer:} Peers exchange traffic between customers
  \includegraphics*[width=8.2cm, height=3.2cm]{images/peertopeer.PNG}
  \begin{itemize}[topsep=0pt,noitemsep,wide=0pt, leftmargin=\dimexpr\labelwidth + 2\labelsep\relax]
    \item AS exports only customer routes to a peer
    \item AS exports a peer’s routes only to its customers
  \end{itemize}
\end{itemize}

\subsubsubsection{\texttt{LOCAL\_PREP} attribute}
\includegraphics*[width=8.2cm, height=3cm]{images/localpref.PNG}
\begin{itemize}[topsep=0pt,noitemsep,wide=0pt, leftmargin=\dimexpr\labelwidth + 2\labelsep\relax]
  \item 4-byte unsigned integer (default value 100)
  \item for a BGP speaker to inform its other internal peers of its degree of preference for a route
  \item should include in UPDATE messages that are sent to internal peers; should not send to external peers
  \item used to coordinate within its \textit{OWN} AS, not transitive, larger value, more this route is preferred
\end{itemize}

\subsubsubsection{\texttt{MULTI\_EXIT\_DISC} attribute}
\includegraphics*[width=8.2cm, height=2cm]{images/multiexecdisc.PNG}
\begin{itemize}[topsep=0pt,noitemsep,wide=0pt, leftmargin=\dimexpr\labelwidth + 2\labelsep\relax]
  \item 4-byte unsigned integer (default value 0)
  \item for a BGP speaker to discriminate among multiple entry points to a neighboring AS to control inbound traffic
  \item if received over eBGP, may be propagated over iBGP, but must not be further propagated to neighboring ASes
\end{itemize}

\subsubsubsection{\texttt{COMMUNITY} attribute}
\begin{itemize}[topsep=0pt,noitemsep,wide=0pt, leftmargin=\dimexpr\labelwidth + 2\labelsep\relax]
  \item Described in RFC 1997 $\rightarrow$ 4-byte integer value
  \item Used to group destinations $\rightarrow$ each destination could be member of multiple communities
  \item Very useful in applying policies within and between ASes
  \item import and export policies based on the \verb|COMMUNITY| attributes
\end{itemize}

\subsubsubsection{BGP Best Route selection}
\begin{enumerate}[topsep=0pt,noitemsep,wide=0pt, leftmargin=\dimexpr\labelwidth + 2\labelsep\relax]
  \item Calculation of degree of preference
  \begin{itemize}[topsep=0pt,noitemsep,wide=0pt, leftmargin=\dimexpr\labelwidth + 2\labelsep\relax]
    \item If the route is learned from an internal peer, use \verb|LOCAL_PREF| attribute or preconfigured policy
    \item Otherwise, use preconfigured policy
  \end{itemize}
  \item Route selection (recommended process)
  \begin{itemize}[topsep=0pt,noitemsep,wide=0pt, leftmargin=\dimexpr\labelwidth + 2\labelsep\relax]
    \item Highest degree of \verb|LOCAL_PREF| (or the only route to the destination), and then tie breaking conditions on:
    \item Smallest number of AS numbers in \verb|AS_PATH| attribute
    \item Lowest origin number in \verb|ORIGIN| attribute
    \item Most preferred \verb|MULTI_EXIT_DISC| attribute
    \item Routes from eBGP are preferred (over iBGP)
    \item Lowest interior cost based on \verb|NEXT_HOP| attribute
  \end{itemize}
\end{enumerate}

\subsubsubsection{BGP Prefix Hijacking}
\includegraphics*[width=8.2cm, height=2.8cm]{images/bgpprefix.PNG}
\begin{itemize}[topsep=0pt,noitemsep,wide=0pt, leftmargin=\dimexpr\labelwidth + 2\labelsep\relax]
  \item Blackhole: data traffic is discarded
  \item Snooping: data traffic is inspected, and then redirected
  \item Impersonation: data traffic is sent to bogus destinations
  \begin{itemize}[topsep=0pt,noitemsep,wide=0pt, leftmargin=\dimexpr\labelwidth + 2\labelsep\relax]
    \item \textbf{SubPrefix Hijacking} Originating a more-specific prefix
    \includegraphics*[width=8cm, height=2.8cm]{images/bgpsubprefix.PNG}
    \item Every AS picks the bogus route for that prefix
    \item Traffic follows the longest matching prefix
  \end{itemize}
  \item \textbf{Preventing SubPrefix Hijacking}
  \begin{itemize}[topsep=0pt,noitemsep,wide=0pt, leftmargin=\dimexpr\labelwidth + 2\labelsep\relax]
    \item Best common practice for route filtering
    \begin{itemize}[topsep=0pt,noitemsep,wide=0pt, leftmargin=\dimexpr\labelwidth + 2\labelsep\relax]
      \item each AS filters routes announced by customers
      \item e.g., based on the prefixes the customer owns
    \end{itemize}
    \item But not everyone applies these practices
    \begin{itemize}[topsep=0pt,noitemsep,wide=0pt, leftmargin=\dimexpr\labelwidth + 2\labelsep\relax]
      \item hard to filter routes initiated from far away
      \item so, BGP remains very vulnerable to hijacks
    \end{itemize}
    \item Industrial trends
    \begin{itemize}[topsep=0pt,noitemsep,wide=0pt, leftmargin=\dimexpr\labelwidth + 2\labelsep\relax]
      \item Route Origin Authorisation (ROA)
      \item Resource Public Key Infrastructure (RPKI)
    \end{itemize}
  \end{itemize}
\end{itemize}

% \pagebreak

% \subsection{A. Miscellaneous}
% \subsubsection{Probability Laws}
% \subsubsubsection{Law of total probability}
% \begin{itemize}[topsep=0pt,noitemsep,wide=0pt, leftmargin=\dimexpr\labelwidth + 2\labelsep\relax]
%   \item \textbf{Discrete distributions:}
%   \begin{itemize}[topsep=0pt,noitemsep,wide=0pt, leftmargin=\dimexpr\labelwidth + 2\labelsep\relax]
%     \item Mutually exclusive events: $E_1, E_2, \ldots , E_n \rightarrow \Sigma_{i=1}^{n} P(E_i) = 1$
%     \item Then $P(E) = \Sigma_{i=1}^{n} P(E | E_i) P (E_i)$
%   \end{itemize}
%   \item \textbf{Continuous distributions:}
%   \begin{itemize}[topsep=0pt,noitemsep,wide=0pt, leftmargin=\dimexpr\labelwidth + 2\labelsep\relax]
%     \item For a random variable $Y \in [0, \infty]$
%     \item Then $P\{X\} = \int_{0}^{\infty} P\{X | Y = y\} f_Y(y)dy$
%   \end{itemize}
% \end{itemize}

% \subsubsubsection{Probability that one event happens before the other}
% Let \textcolor{blue}{$X$} and \textcolor{red}{$Y$} be 2 independent exponential random variables with parameters $\textcolor{blue}{\lambda_x}$ and $\textcolor{red}{\lambda_y}$. What is the probability that $P\{\textcolor{blue}{X} > \textcolor{red}{Y}\}$
% \[
%   \begin{aligned}
%     P\{\textcolor{blue}{X} > \textcolor{red}{Y}\} &= \int_{0}^{\infty} P\{\textcolor{blue}{X} > \textcolor{red}{Y} | \textcolor{red}{Y} = y\} = f_{\textcolor{red}{Y}}(y)dy \\ 
%     &= \int_{0}^{\infty} P\{X > y\}(\textcolor{red}{\lambda_y} e^{-\textcolor{red}{\lambda_y} y})dy = \int_{0}^{\infty}e^{\textcolor{blue}{\lambda_x} y} (\textcolor{red}{\lambda_y} e^{-\textcolor{red}{\lambda_y} y})dy \\ 
%     &= \textcolor{red}{\lambda_y} \int_{0}^{\infty}e^{-(\textcolor{blue}{\lambda_x} + \textcolor{red}{\lambda_y})y}dy = \textcolor{red}{\lambda_y} (-\frac{1}{\textcolor{blue}{\lambda_x} + \textcolor{red}{\lambda_y}} e^{-(\textcolor{blue}{\lambda_x} + \textcolor{red}{\lambda_y})y} |^{\infty}_0) \\ 
%     &= \frac{\textcolor{red}{\lambda_y}}{\textcolor{blue}{\lambda_x} + \textcolor{red}{\lambda_y}}
%   \end{aligned}
% \]

% \subsubsection*{Summations}
% \subsubsubsection{Geometric Series}
% \begin{itemize}[topsep=0pt,noitemsep,wide=0pt, leftmargin=\dimexpr\labelwidth + 2\labelsep\relax]
%   \item $S_n = \frac{a(1-r^n)}{1-r}$ | $S_\infty = \frac{a}{1-r}$
%   \begin{itemize}[topsep=0pt,noitemsep,wide=0pt, leftmargin=\dimexpr\labelwidth + 2\labelsep\relax]
%     \item where $a$ is the first term, $r$ is the common ratio, $n$ is number of terms
%   \end{itemize}
% \end{itemize}
% \subsubsubsection{Arithmetic Series}
% \begin{itemize}[topsep=0pt,noitemsep,wide=0pt, leftmargin=\dimexpr\labelwidth + 2\labelsep\relax]
%   \item $S_n = n (\frac{a_1 + a_n}{2})$, $n$ is number of terms, $a_1$ is first term, $a_n$ is $n^{th}$ term
% \end{itemize}


\end{multicols*}

\end{document}